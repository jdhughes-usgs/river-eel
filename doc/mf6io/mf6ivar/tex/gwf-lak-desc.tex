% DO NOT MODIFY THIS FILE DIRECTLY.  IT IS CREATED BY mf6ivar.py 

\item \texttt{auxiliary}---defines an array of one or more auxiliary variable names.  There is no limit on the number of auxiliary variables that can be provided on this line; however, lists of information provided in subsequent blocks must have a column of data for each auxiliary variable name defined here.   The number of auxiliary variables detected on this line determines the value for naux.  Comments cannot be provided anywhere on this line as they will be interpreted as auxiliary variable names.  Auxiliary variables may not be used by the package, but they will be available for use by other parts of the program.  The program will terminate with an error if auxiliary variables are specified on more than one line in the options block.

\item \texttt{BOUNDNAMES}---keyword to indicate that boundary names may be provided with the list of lake cells.

\item \texttt{PRINT\_INPUT}---keyword to indicate that the list of lake information will be written to the listing file immediately after it is read.

\item \texttt{PRINT\_STAGE}---keyword to indicate that the list of lake stages will be printed to the listing file for every stress period in which ``HEAD PRINT'' is specified in Output Control.  If there is no Output Control option and \texttt{PRINT\_STAGE} is specified, then stages are printed for the last time step of each stress period.

\item \texttt{PRINT\_FLOWS}---keyword to indicate that the list of lake flow rates will be printed to the listing file for every stress period time step in which ``BUDGET PRINT'' is specified in Output Control.  If there is no Output Control option and \texttt{PRINT\_FLOWS} is specified, then flow rates are printed for the last time step of each stress period.

\item \texttt{SAVE\_FLOWS}---keyword to indicate that lake flow terms will be written to the file specified with ``BUDGET FILEOUT'' in Output Control.

\item \texttt{STAGE}---keyword to specify that record corresponds to stage.

\item \texttt{stagefile}---name of the binary output file to write stage information.

\item \texttt{BUDGET}---keyword to specify that record corresponds to the budget.

\item \texttt{FILEOUT}---keyword to specify that an output filename is expected next.

\item \texttt{budgetfile}---name of the binary output file to write budget information.

\item \texttt{TS6}---keyword to specify that record corresponds to a time-series file.

\item \texttt{FILEIN}---keyword to specify that an input filename is expected next.

\item \texttt{ts6\_filename}---defines a time-series file defining time series that can be used to assign time-varying values. See the ``Time-Variable Input'' section for instructions on using the time-series capability.

\item \texttt{OBS6}---keyword to specify that record corresponds to an observations file.

\item \texttt{obs6\_filename}---name of input file to define observations for the LAK package. See the ``Observation utility'' section for instructions for preparing observation input files. Table \ref{table:obstype} lists observation type(s) supported by the LAK package.

\item \texttt{MOVER}---keyword to indicate that this instance of the LAK Package can be used with the Water Mover (MVR) Package.  When the \texttt{MOVER} option is specified, additional memory is allocated within the package to store the available, provided, and received water.

\item \texttt{surfdep}---real value that defines the surface depression depth for \texttt{VERTICAL} lake-\texttt{GWF} connections. If specified, \texttt{surfdep} must be greater than or equal to zero. If \texttt{SURFDEP} is not specified, a default value of zero is used for all vertical lake-\texttt{GWF} connections.

\item \texttt{time\_conversion}---value that is used in converting outlet flow terms that use Manning's equation or gravitational acceleration to consistent time units. \texttt{time\_conversion} should be set to 1.0, 60.0, 3,600.0, 86,400.0, and 31,557,600.0 when using time units (\texttt{time\_units}) of seconds, minutes, hours, days, or years in the simulation, respectively. \texttt{convtime} does not need to be specified if no lake outlets are specified or \texttt{time\_units} are seconds.

\item \texttt{length\_conversion}---real value that is used in converting outlet flow terms that use Manning's equation or gravitational acceleration to consistent length units. \texttt{length\_conversion} should be set to 3.28081, 1.0, and 100.0 when using length units (\texttt{length\_units}) of feet, meters, or centimeters in the simulation, respectively. \texttt{length\_conversion} does not need to be specified if no lake outlets are specified or \texttt{length\_units} are meters.

\item \texttt{nlakes}---value specifying the number of lakes that will be simulated for all stress periods.

\item \texttt{noutlets}---value specifying the number of outlets that will be simulated for all stress periods. If \texttt{NOUTLETS} is not specified, a default value of zero is used.

\item \texttt{ntables}---value specifying the number of lakes tables that will be used to define the lake stage, volume relation, and surface area. If \texttt{NTABLES} is not specified, a default value of zero is used.

\item \texttt{lakeno}---integer value that defines the lake number for this lake entry. \texttt{lakeno} must be greater than zero and less than or equal to \texttt{nlakes}.

\item \texttt{strt}---real value that defines the starting stage for the lake.

\item \texttt{nlakeconn}---integer value that defines the number of \texttt{GWF} nodes connected to this (\texttt{lakeno}) lake. There can only be one vertical lake connection to each \texttt{GWF} node. \texttt{nlakeconn} must be greater than zero.

\item \textcolor{blue}{\texttt{aux}---represents the values of the auxiliary variables for each lake. The values of auxiliary variables must be present for each lake. The values must be specified in the order of the auxiliary variables specified in the OPTIONS block.  If the package supports time series and the Options block includes a TIMESERIESFILE entry (see the ``Time-Variable Input'' section), values can be obtained from a time series by entering the time-series name in place of a numeric value.}

\item \texttt{boundname}---name of the lake cell.  \texttt{boundname} is an ASCII character variable that can contain as many as 40 characters.  If \texttt{boundname} contains spaces in it, then the entire name must be enclosed within single quotes.

\item \texttt{lakeno}---integer value that defines the lake number for this lake entry. \texttt{lakeno} must be greater than zero and less than or equal to \texttt{nlakes}.

\item \texttt{iconn}---integer value that defines the \texttt{GWF} connection number for this lake connection entry. \texttt{iconn} must be greater than zero and less than or equal to \texttt{nlakeconn} for lake \texttt{lakeno}.

\item \texttt{cellid}---is the cell identifier, and depends on the type of grid that is used for the simulation.  For a structured grid that uses the DIS input file, \texttt{cellid} is the layer, row, and column.   For a grid that uses the DISV input file, \texttt{cellid} is the layer and cell2d number.  If the model uses the unstructured discretization (DISU) input file, then \texttt{cellid} is the node number for the cell.

\item \texttt{claktype}---character string that defines the lake-\texttt{GWF} connection type for the lake connection. Possible lake-\texttt{GWF} connection type strings include:  \texttt{VERTICAL}--character keyword to indicate the lake-\texttt{GWF} connection is vertical  and connection conductance calculations use the hydraulic conductivity corresponding to the $K_{33}$ tensor component defined for \texttt{cellid} in the NPF package. \texttt{HORIZONTAL}--character keyword to indicate the lake-\texttt{GWF} connection is horizontal and connection conductance calculations use the hydraulic conductivity corresponding to the $K_{11}$ tensor component defined for \texttt{cellid} in the NPF package. \texttt{EMBEDDEDH}--character keyword to indicate the lake-\texttt{GWF} connection is embedded in a single cell and connection conductance calculations use the hydraulic conductivity corresponding to the $K_{11}$ tensor component defined for \texttt{cellid} in the NPF package. \texttt{EMBEDDEDV}--character keyword to indicate the lake-\texttt{GWF} connection is embedded in a single cell and connection conductance calculations use the hydraulic conductivity corresponding to the $K_{33}$ tensor component defined for \texttt{cellid} in the NPF package. Embedded lakes can only be connected to a single cell (\texttt{nlakconn = 1}) and there must be a lake table associated with each embedded lake.

\item \texttt{bedleak}---real value that defines the bed leakance for the lake-\texttt{GWF} connection. \texttt{bedk} must be greater than or equal to zero.

\item \texttt{belev}---real value that defines the bottom elevation for a \texttt{HORIZONTAL} lake-\texttt{GWF} connection. Any value can be specified if \texttt{claktype} is \texttt{VERTICAL}, \texttt{EMBEDDEDH}, or \texttt{EMBEDDEDV}. If \texttt{claktype} is \texttt{HORIZONTAL} and \texttt{belev} is not equal to \texttt{telev}, \texttt{belev} must be greater than or equal to the bottom of the \texttt{GWF} cell \texttt{cellid}. If \texttt{belev} is equal to \texttt{telev}, \texttt{belev} is reset to the bottom of the \texttt{GWF} cell \texttt{cellid}.

\item \texttt{telev}---real value that defines the top elevation for a \texttt{HORIZONTAL} lake-\texttt{GWF} connection. Any value can be specified if \texttt{claktype} is \texttt{VERTICAL}, \texttt{EMBEDDEDH}, or \texttt{EMBEDDEDV}. If \texttt{claktype} is \texttt{HORIZONTAL} and \texttt{telev} is not equal to \texttt{belev}, \texttt{telev} must be less than or equal to the top of the \texttt{GWF} cell \texttt{cellid}. If \texttt{telev} is equal to \texttt{belev}, \texttt{telev} is reset to the top of the \texttt{GWF} cell \texttt{cellid}.

\item \texttt{connlen}---real value that defines the distance between the connected \texttt{GWF} \texttt{cellid} node and the lake for a \texttt{HORIZONTAL}, \texttt{EMBEDDEDH}, or \texttt{EMBEDDEDV} lake-\texttt{GWF} connection. \texttt{connlen} must be greater than zero for a \texttt{HORIZONTAL}, \texttt{EMBEDDEDH}, or \texttt{EMBEDDEDV} lake-\texttt{GWF} connection. Any value can be specified if \texttt{claktype} is \texttt{VERTICAL}.

\item \texttt{connwidth}---real value that defines the connection face width for a \texttt{HORIZONTAL} lake-\texttt{GWF} connection. \texttt{connwidth} must be greater than zero for a \texttt{HORIZONTAL} lake-\texttt{GWF} connection. Any value can be specified if \texttt{claktype} is \texttt{VERTICAL}, \texttt{EMBEDDEDH}, or \texttt{EMBEDDEDV}.

\item \texttt{lakeno}---integer value that defines the lake number for this lake entry. \texttt{lakeno} must be greater than zero and less than or equal to \texttt{nlakes}.

\item \texttt{TAB6}---keyword to specify that record corresponds to a table file.

\item \texttt{FILEIN}---keyword to specify that an input filename is expected next.

\item \texttt{tab6\_filename}---character string that defines the path and filename for the file containing lake table data for the lake connection. The \texttt{ctabname} file includes the number of entries in the file and the relation between stage, surface area, and volume for each entry in the file. Lake table files for \texttt{EMBEDDEDH} and \texttt{EMBEDDEDV} lake-\texttt{GWF} connections also include lake-\texttt{GWF} exchange area data for each entry in the file. Input instructions for the \texttt{ctabname} file is included at the LAK package lake table file input instructions section.

\item \texttt{outletno}---integer value that defines the outlet number for this outlet entry. \texttt{outletno} must be greater than zero and less than or equal to \texttt{noutlets}.

\item \texttt{lakein}---integer value that defines the lake number that outlet is connected to. \texttt{lakein} must be greater than zero and less than or equal to \texttt{nlakes}.

\item \texttt{lakeout}---integer value that defines the lake number that outlet discharge from lake outlet \texttt{outletno} is routed to. \texttt{lakeout} must be greater than or equal to zero and less than or equal to \texttt{nlakes}. If \texttt{lakeout} is zero, outlet discharge from lake outlet \texttt{outletno} is discharged to an external boundary.

\item \texttt{couttype}---character string that defines the outlet type for the outlet \texttt{outletno}. Possible \texttt{couttype} strings include: \texttt{SPECIFIED}--character keyword to indicate the outlet is defined as a specified flow.  \texttt{MANNING}--character keyword to indicate the outlet is defined using Manning's equation. \texttt{WEIR}--character keyword to indicate the outlet is defined using a sharp weir equation.

\item \textcolor{blue}{\texttt{invert}---real value that defines the invert elevation for the lake outlet. Any value can be specified if \texttt{couttype} is \texttt{SPECIFIED}. If the Options block includes a \texttt{TIMESERIESFILE} entry (see the ``Time-Variable Input'' section), values can be obtained from a time series by entering the time-series name in place of a numeric value.}

\item \textcolor{blue}{\texttt{width}---real value that defines the width of the lake outlet. Any value can be specified if \texttt{couttype} is \texttt{SPECIFIED}. If the Options block includes a \texttt{TIMESERIESFILE} entry (see the ``Time-Variable Input'' section), values can be obtained from a time series by entering the time-series name in place of a numeric value.}

\item \textcolor{blue}{\texttt{rough}---real value that defines the roughness coefficient for the lake outlet. Any value can be specified if \texttt{couttype} is not \texttt{MANNING}. If the Options block includes a \texttt{TIMESERIESFILE} entry (see the ``Time-Variable Input'' section), values can be obtained from a time series by entering the time-series name in place of a numeric value.}

\item \textcolor{blue}{\texttt{slope}---real value that defines the bed slope for the lake outlet. Any value can be specified if \texttt{couttype} is not \texttt{MANNING}. If the Options block includes a \texttt{TIMESERIESFILE} entry (see the ``Time-Variable Input'' section), values can be obtained from a time series by entering the time-series name in place of a numeric value.}

\item \texttt{iper}---integer value specifying the starting stress period number for which the data specified in the PERIOD block apply.  \texttt{iper} must be less than \texttt{nper} in the TDIS Package and greater than zero.  The \texttt{iper} value assigned to a stress period block must be greater than the \texttt{iper} value assigned for the previous block.

\item \texttt{lakeno}---integer value that defines the lake number associated with the specified data on the line. \texttt{lakeno} must be greater than zero and less than or equal to \texttt{nlakes}.

\item \texttt{laksetting}---line of information that is parsed into a keyword and values.  Keyword values that can be used to start the \texttt{laksetting} string include: \texttt{STATUS}, \texttt{STAGE}, \texttt{RAINFALL}, \texttt{EVAPORATION}, \texttt{RUNOFF}, \texttt{WITHDRAWAL}, and \texttt{AUXILIARY}.

\begin{lstlisting}[style=blockdefinition]
STATUS <status>
STAGE <@stage@>
RAINFALL <@rainfall@>
EVAPORATION <@evaporation@>
RUNOFF <@runoff@>
WITHDRAWAL <@withdrawal@>
AUXILIARY <auxname> <@auxval@> 
\end{lstlisting}

\item \texttt{status}---keyword option to define lake status.  \texttt{status} can be \texttt{ACTIVE}, \texttt{INACTIVE}, or \texttt{CONSTANT}. By default, \texttt{status} is \texttt{ACTIVE}.

\item \textcolor{blue}{\texttt{rate}---real or character value that defines the extraction rate for the lake outflow. A positive value indicates inflow and a negative value indicates outflow from the lake. \texttt{rate} only applies to active (\texttt{IBOUND}$>0$) lakes. A specified \texttt{rate} is only applied if \texttt{couttype} for the \texttt{outletno} is \texttt{SPECIFIED}. If the Options block includes a \texttt{TIMESERIESFILE} entry (see the ``Time-Variable Input'' section), values can be obtained from a time series by entering the time-series name in place of a numeric value. By default, the \texttt{rate} for each \texttt{SPECIFIED} lake outlet is zero.}

\item \textcolor{blue}{\texttt{stage}---real or character value that defines the stage for the lake. The specified \texttt{stage} is only applied if the lake is a constant stage lake. If the Options block includes a \texttt{TIMESERIESFILE} entry (see the ``Time-Variable Input'' section), values can be obtained from a time series by entering the time-series name in place of a numeric value.}

\item \textcolor{blue}{\texttt{rainfall}---real or character value that defines the rainfall rate for the lake. \texttt{value} must be greater than or equal to zero. If the Options block includes a \texttt{TIMESERIESFILE} entry (see the ``Time-Variable Input'' section), values can be obtained from a time series by entering the time-series name in place of a numeric value.}

\item \textcolor{blue}{\texttt{evaporation}---real or character value that defines the maximum evaporation rate for the lake. \texttt{value} must be greater than or equal to zero. If the Options block includes a \texttt{TIMESERIESFILE} entry (see the ``Time-Variable Input'' section), values can be obtained from a time series by entering the time-series name in place of a numeric value.}

\item \textcolor{blue}{\texttt{runoff}---real or character value that defines the runoff rate for the lake. \texttt{value} must be greater than or equal to zero. If the Options block includes a \texttt{TIMESERIESFILE} entry (see the ``Time-Variable Input'' section), values can be obtained from a time series by entering the time-series name in place of a numeric value.}

\item \textcolor{blue}{\texttt{withdrawal}---real or character value that defines the maximum withdrawal rate for the lake. \texttt{value} must be greater than or equal to zero. If the Options block includes a \texttt{TIMESERIESFILE} entry (see the ``Time-Variable Input'' section), values can be obtained from a time series by entering the time-series name in place of a numeric value.}

\item \texttt{AUXILIARY}---keyword for specifying auxiliary variable.

\item \texttt{auxname}---name for the auxiliary variable to be assigned \texttt{auxval}.  \texttt{auxname} must match one of the auxiliary variable names defined in the \texttt{OPTIONS} block. If \texttt{auxname} does not match one of the auxiliary variable names defined in the \texttt{OPTIONS} block the data are ignored.

\item \textcolor{blue}{\texttt{auxval}---value for the auxiliary variable. If the Options block includes a \texttt{TIMESERIESFILE} entry (see the ``Time-Variable Input'' section), values can be obtained from a time series by entering the time-series name in place of a numeric value.}

\item \texttt{outletno}---integer value that defines the outlet number for this outlet entry. \texttt{outletno} must be greater than zero and less than or equal to \texttt{noutlets}.

\item \texttt{outletsetting}---line of information that is parsed into a keyword and values.  Keyword values that can be used to start the \texttt{outletsetting} string include: \texttt{RATE}, \texttt{INVERT}, \texttt{WIDTH}, \texttt{SLOPE}, and \texttt{ROUGH}.

\begin{lstlisting}[style=blockdefinition]
RATE <@rate@>
INVERT <@invert@>
WIDTH <@width@>
SLOPE <@slope@>
ROUGH <@rough@>
\end{lstlisting}

\item \textcolor{blue}{\texttt{invert}---real or character value that defines the invert elevation for the lake outlet. A specified \texttt{invert} value is only used for active lakes if \texttt{couttype} for lake outlet \texttt{outletno} is not \texttt{SPECIFIED}. If the Options block includes a \texttt{TIMESERIESFILE} entry (see the ``Time-Variable Input'' section), values can be obtained from a time series by entering the time-series name in place of a numeric value.}

\item \textcolor{blue}{\texttt{rough}---real or character value that defines the width of the lake outlet. A specified \texttt{width} value is only used for active lakes if \texttt{couttype} for lake outlet \texttt{outletno} is not \texttt{SPECIFIED}. If the Options block includes a \texttt{TIMESERIESFILE} entry (see the ``Time-Variable Input'' section), values can be obtained from a time series by entering the time-series name in place of a numeric value.}

\item \textcolor{blue}{\texttt{width}---real or character value that defines the width of the lake outlet. A specified \texttt{width} value is only used for active lakes if \texttt{couttype} for lake outlet \texttt{outletno} is not \texttt{SPECIFIED}. If the Options block includes a \texttt{TIMESERIESFILE} entry (see the ``Time-Variable Input'' section), values can be obtained from a time series by entering the time-series name in place of a numeric value.}

\item \textcolor{blue}{\texttt{slope}---real or character value that defines the bed slope for the lake outlet. A specified \texttt{slope} value is only used for active lakes if \texttt{couttype} for lake outlet \texttt{outletno} is \texttt{MANNING}. If the Options block includes a \texttt{TIMESERIESFILE} entry (see the ``Time-Variable Input'' section), values can be obtained from a time series by entering the time-series name in place of a numeric value.}


